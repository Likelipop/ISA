\section{Data preparation}

Data preparation is one of the most time-consuming phrase in comparison with others.
It comprises several activities for getting these data ready for later analytics phrases.
typically, a data preparation consists of:
\begin{itemize}
    \item \textbf{Collect data}: the whole data was collected for us so we mainly loads it into our workspace and use
    \item \textbf{Cleaning data}: publisher once claims the data was clean. But we still examines some of the importance criteria such as missing values, duplicated data and outliers.
    when it comes to imbalance data, we processed it later in the modeling sections.
    \item \textbf{Exploratory Data analysis}: this involved 
    summarizing the main characteristics of the data to gain better insights and 
    validate assumptions. We modified this section becomes the \textbf{Exploration and Visualization}
    so that we can emphrasize their characteristics and relationship between them later.
\end{itemize}


\subsection{Data Overview}

\subsubsection{Identification of Variable Types}

Variables were classified into numeric and categorical types based on their underlying measurement scales. Numeric variables were defined as continuous measures with meaningful quantitative interpretation, while categorical variables included nominal and ordinal attributes representing discrete or ordered categories.

\begin{table}[H]
\centering
\caption{Variable Types in the Dataset}
\begin{tabular}{p{6cm}|p{9cm}}
\hline
\textbf{Numeric Variables} & \textbf{Categorical Variables} \\
\hline
BMI, MentHlth, PhysHlth &
Diabetes\_012, HighBP, HighChol, CholCheck, Smoker, Stroke, HeartDiseaseorAttack, \\
& PhysActivity, Fruits, Veggies, HvyAlcoholConsump, AnyHealthcare, NoDocbcCost, \\
& GenHlth, DiffWalk, Sex, Age, Education, Income \\
\hline
\end{tabular}
\end{table}

Although the Age variable is numerically coded in the BRFSS dataset, it represents ordinal age groups rather than continuous age measured in years. Since the numeric codes do not correspond to equal intervals, treating Age as a continuous variable could lead to misleading statistical inference. Therefore, Age was treated as a categorical variable in subsequent analyses.

This classification strategy was adopted to avoid statistical bias and ensure valid interpretation in descriptive statistics and predictive modeling.

\subsubsection{Descriptive Statistics for Numeric Variables}

Descriptive statistics were used to summarize numeric variables and assess their distributional characteristics.

\begin{table}[H]
\centering
\caption{Descriptive statistics for numeric variables}
\label{tab:desc_numeric}

\resizebox{0.8\textwidth}{!}{%
\begin{tabular}{|l|r|r|r|r|r|r|}
\hline
Variable & Mean & SD & Median & IQR & Min & Max \\
\hline
BMI      & 28.382364 & 6.608694 & 27 & 7 & 12 & 98 \\
MentHlth & 3.184772  & 7.412847 & 0  & 2 & 0  & 30 \\
PhysHlth & 4.242081  & 8.717951 & 0  & 3 & 0  & 30 \\
\hline
\end{tabular}
}
\end{table}

The BMI variable exhibits a relatively high mean value along with a large standard deviation, indicating substantial variability in body weight within the study population. This reflects the coexistence of multiple weight groups, including a considerable proportion of individuals classified as overweight or obese.

In contrast, MentHlth and PhysHlth display asymmetric distributions with median values equal to zero and long right tails, indicating that most respondents did not experience days of poor health, while a small subgroup reported prolonged periods of poor physical or mental health. These distributional characteristics reflect heterogeneity in self-reported health status and suggest that these variables may play an important role in subsequent inferential analyses.

\subsubsection{Descriptive Statistics for Categorical Variables}

Descriptive statistics were used to summarize categorical variables and assess their distributional characteristics.

\begin{table}[H]
\centering
\caption{Descriptive statistics for categorical variables}
\label{tab:desc_categorical}

\footnotesize
\setlength{\tabcolsep}{5pt}
\renewcommand{\arraystretch}{1.1}

\resizebox{0.9\textwidth}{!}{%
\begin{tabular}{|l|r|r|r|r|}
\hline
\textbf{Variable} & \textbf{n\_levels} & \textbf{Mode} & \textbf{Mode Count} & \textbf{Mode (\%)} \\
\hline
Age                & 13 & 9 & 33244  & 13.10 \\
AnyHealthcare      & 2  & 1 & 241263 & 95.11 \\
CholCheck           & 2  & 1 & 244210 & 96.27 \\
Diabetes\_012        & 3  & 0 & 213703 & 84.24 \\
DiffWalk            & 2  & 0 & 211005 & 83.18 \\
Education           & 6  & 6 & 107325 & 42.31 \\
Fruits              & 2  & 1 & 160898 & 63.43 \\
GenHlth             & 5  & 2 & 89084  & 35.12 \\
HeartDiseaseorAttack& 2  & 0 & 229787 & 90.58 \\
HighBP              & 2  & 0 & 144851 & 57.10 \\
HighChol             & 2  & 0 & 146089 & 57.59 \\
HvyAlcoholConsump    & 2  & 0 & 239424 & 94.38 \\
Income               & 8  & 8 & 90385  & 35.63 \\
NoDocbcCost           & 2  & 0 & 233326 & 91.58 \\
PhysActivity          & 2  & 1 & 191920 & 75.65 \\
Sex                   & 2  & 0 & 141974 & 55.97 \\
Smoker                & 2  & 0 & 141257 & 55.68 \\
Stroke                & 2  & 0 & 243388 & 95.94 \\
Veggies               & 2  & 1 & 205841 & 81.14 \\
\hline
\end{tabular}
}
\end{table}

Most categorical variables exhibit class imbalance, particularly those related to health status and healthcare services. In contrast, demographic and socioeconomic variables show more evenly distributed class structures, reflecting greater heterogeneity across population groups. Therefore, class imbalance should be carefully considered in subsequent analyses and modeling tasks.

\section{Data Preprocessing}
\subsection{Duplicate Data Assessment}

Duplicate data assessment was conducted to ensure data reliability and to mitigate potential biases in subsequent analyses.

\textbf{Result:} \\
The duplicate data assessment identified \textbf{23{,}899} fully duplicated observations, accounting for approximately \textbf{9.42\%} of the total dataset.

\textbf{Conclusion:} \\
Although a moderate proportion of duplicated records was detected, these observations were not removed from the dataset. In large-scale survey data, identical records can correspond to different individuals who share the same demographic and health profiles. Eliminating such records could artificially alter the population structure and reduce the statistical power of subsequent analyses. Therefore, duplicated observations were retained to maintain the integrity and representativeness of the dataset.

\subsection{Missing Data Assessment}

Missing data assessment was performed to evaluate the extent and patterns of missingness in the dataset.
 
\begin{table}[H]
\centering
\caption{Missing data patterns}
\label{tab:missing_data}

\footnotesize
\setlength{\tabcolsep}{5pt}
\renewcommand{\arraystretch}{1.1}

\begin{tabular}{|l|r|r|}
\hline
\textbf{Variable} & \textbf{Missing Count} & \textbf{Missing Rate} \\
\hline
Diabetes\_012          & 0 & 0 \\
HighBP                 & 0 & 0 \\
HighChol               & 0 & 0 \\
CholCheck              & 0 & 0 \\
BMI                    & 0 & 0 \\
Smoker                 & 0 & 0 \\
Stroke                 & 0 & 0 \\
HeartDiseaseorAttack   & 0 & 0 \\
PhysActivity            & 0 & 0 \\
Fruits                  & 0 & 0 \\
Veggies                 & 0 & 0 \\
HvyAlcoholConsump        & 0 & 0 \\
AnyHealthcare            & 0 & 0 \\
NoDocbcCost              & 0 & 0 \\
GenHlth                  & 0 & 0 \\
MentHlth                 & 0 & 0 \\
PhysHlth                 & 0 & 0 \\
DiffWalk                 & 0 & 0 \\
Sex                      & 0 & 0 \\
Age                      & 0 & 0 \\
Education                & 0 & 0 \\
Income                   & 0 & 0 \\
\hline
\end{tabular}
\end{table}

\textbf{Conclusion:} \\
The missing data assessment indicates that no missing values were detected across all variables in the dataset. Therefore, no missing data handling or imputation techniques were required.

\subsection{Outlier Assessment}

Outlier assessment was conducted to identify extreme values that may affect statistical inference and predictive modeling. Density plots were used to examine distributional patterns, while boxplots based on the interquartile range (IQR) criterion were applied to detect potential outliers. These methods help distinguish data anomalies from meaningful extreme observations and support appropriate outlier handling decisions.

\subsubsection{BMI}
\begin{figure}[H]
\centering
\includegraphics[width=1\textwidth]{img/BMI_density.png}
\caption{Density plot of BMI distribution}
\label{fig:bmi_density}
\end{figure}

\begin{figure}[H]
\centering
\includegraphics[width=1\textwidth]{img/BMI_boxplot.png}
\caption{Boxplot of BMI}
\label{fig:bmi_boxplot}
\end{figure}

The boxplot and density plot of BMI indicate the presence of numerous high-end extreme values. The boxplot shows many observations located above the upper whisker, suggesting potential outliers at the upper tail of the distribution. Consistently, the density plot reveals a strongly right-skewed distribution with a long right tail, indicating that a small proportion of individuals have extremely high BMI values. These extreme observations reflect substantial heterogeneity in body mass within the population and may have a considerable influence on statistical analyses and modeling results.

\textbf{Outlier Case Study:} \\
After applying density plots and IQR-based boxplots to identify outliers in BMI, the study further classified BMI according to the World Health Organization (WHO) categories, including underweight (BMI < 18.5), normal weight (18.5 ≤ BMI < 25), overweight (25 ≤ BMI < 30), and obesity (BMI ≥ 30).

To assess the impact of extreme BMI values, several representative observations with exceptionally high BMI were selected for in-depth analysis. Detailed information on these cases are presented in Table~\ref{tab:bmi_cases}.

\begin{table}[H]
\centering
\caption{Selected individuals with different BMI levels}
\label{tab:bmi_cases}

\scriptsize
\setlength{\tabcolsep}{4pt}
\renewcommand{\arraystretch}{1.1}

\begin{tabular}{l|c|c|c|c}
\hline
\textbf{Variable} & \textbf{Case 1 (BMI=40)} & \textbf{Case 2 (BMI=63)} & \textbf{Case 3 (BMI=85)} & \textbf{Case 4 (BMI=98)} \\
\hline
Diabetes\_012 & 0 & 1 & 2 & 2 \\
HighBP & 1 & 1 & 1 & 1 \\
HighChol & 1 & 1 & 1 & 1 \\
CholCheck & 1 & 1 & 1 & 1 \\
BMI & 40 & 63 & 85 & 98 \\
Smoker & 0 & 0 & 1 & 1 \\
Stroke & 0 & 1 & 1 & 0 \\
HeartDiseaseorAttack & 0 & 1 & 1 & 1 \\
PhysActivity & 0 & 0 & 0 & 0 \\
Fruits & 0 & 1 & 0 & 1 \\
Veggies & 1 & 1 & 0 & 1 \\
HvyAlcoholConsump & 0 & 0 & 0 & 0 \\
AnyHealthcare & 1 & 1 & 1 & 1 \\
NoDocbcCost & 0 & 0 & 0 & 0 \\
GenHlth & 5 & 4 & 5 & 5 \\
MentHlth & 18 & 0 & 0 & 30 \\
PhysHlth & 15 & 30 & 30 & 30 \\
DiffWalk & 1 & 1 & 1 & 1 \\
Sex & 0 & 0 & 0 & 0 \\
Age & 9 & 6 & 6 & 8 \\
Education & 4 & 4 & 4 & 6 \\
Income & 3 & 2 & 1 & 5 \\
\hline
\end{tabular}
\end{table}

From an epidemiological perspective, a high BMI not only reflects excessive adipose tissue accumulation but is also strongly associated with metabolic disorders, particularly hypertension and dyslipidemia. Biological mechanisms such as insulin resistance, chronic low-grade inflammation, and adipose tissue endocrine dysfunction have been identified as key pathways linking obesity to an increased risk of diabetes. Previous cohort studies have shown that central obesity and other components of the metabolic syndrome, including hypertension and dyslipidemia, are significantly associated with an elevated risk of type 2 diabetes.

(Cite: https://pmc.ncbi.nlm.nih.gov/articles/PMC5023917/)

In the observed case, the coexistence of extremely high BMI with hypertension and high cholesterol is consistent with the metabolic risk profile commonly reported among obese individuals. This suggests that the extreme BMI value is not a data anomaly, but rather an extreme yet plausible manifestation within the population distribution that carries important information regarding cardiometabolic disease risk. Therefore, very high BMI values play a critical role in evaluating the relationship between obesity and diabetes risk and should not be removed as statistical outliers.

\begin{figure}[H]
\centering
\includegraphics[width=0.7\textwidth]{img/BMI_category.png}
\caption{Distribution of BMI categories based on WHO classification}
\label{fig:bmi_categories}
\end{figure}

Figure~\ref{fig:bmi_categories} shows that a large proportion of individuals fall into the overweight and obese categories, reflecting a substantial burden of obesity in the study population. If extreme observations were removed, the BMI distribution would be artificially truncated, leading to an underestimation of the role of severe obesity and a distorted assessment of the true relationship between BMI and diabetes risk. These extreme cases represent the highest-risk group and contain important information about disease mechanisms.

\textbf{Conclusion:} \\
Based on visual evidence and case study analysis, this study decided to retain BMI outliers to ensure data representativeness and maintain the reliability of conclusions regarding the role of obesity in diabetes risk. Retaining extreme values allows the data to fully reflect real-world population variability and prevents bias in statistical analyses and predictive modeling.

\subsubsection{Physical Health (PhysHlth)}

\begin{figure}[H]
\centering
\includegraphics[width=1\textwidth]{img/PhysHlth_density.png}
\caption{Density plot of Physical Health distribution}
\label{fig:phys_density}
\end{figure}

\begin{figure}[H]
\centering
\includegraphics[width=1\textwidth]{img/PhysHlth_boxplot.png}
\caption{Boxplot of Physical Health}
\label{fig:phys_boxplot}
\end{figure}

The boxplot and density plot of PhysHlth show a strongly right-skewed distribution with many high-end extreme values. The boxplot indicates several observations beyond the upper whisker, while the density plot exhibits a sharp peak near zero and a long right tail. This suggests that most respondents reported few poor physical health days, but a small group experienced extremely high values, which may significantly affect statistical analysis.

\textbf{Outlier Case Study:} \\
After using density plots and IQR-based boxplots to identify outliers in the \textit{PhysHlth} variable, the study further examined the epidemiological relevance of these extreme observations. To improve interpretability and reduce data sparsity at higher values, \textit{PhysHlth} was categorized into four ordinal levels (0 days, 1--7 days, 8--14 days, and 15--30 days), representing increasing levels of physical health impairment over the past 30 days.

To assess the impact of extreme values, several representative observations in the 15–30 day category were selected for in-depth analysis. Detailed information on these cases is presented in Table~\ref{tab:physhlth_cases}.
\begin{table}[H]
\centering
\caption{Selected diabetic individuals with high PhysHlth values (case studies)}
\label{tab:physhlth_cases}

\scriptsize
\setlength{\tabcolsep}{4pt}
\renewcommand{\arraystretch}{1.1}

\resizebox{\textwidth}{!}{%
\begin{tabular}{l|c|c|c|c}
\hline
\textbf{Variable} & \textbf{Case 1 (PhysHlth=30)} & \textbf{Case 2 (PhysHlth=26)} & \textbf{Case 3 (PhysHlth=21)} & \textbf{Case 4 (PhysHlth=20)} \\
\hline
Diabetes\_012 & 1 & 1 & 1 & 1 \\
HighBP & 1 & 1 & 1 & 0 \\
HighChol & 1 & 0 & 1 & 1 \\
CholCheck & 1 & 1 & 1 & 1 \\
BMI & 37 & 36 & 32 & 45 \\
Smoker & 0 & 1 & 1 & 1 \\
Stroke & 1 & 0 & 0 & 0 \\
HeartDiseaseorAttack & 0 & 1 & 0 & 0 \\
PhysActivity & 1 & 0 & 1 & 0 \\
Fruits & 0 & 0 & 1 & 1 \\
Veggies & 1 & 0 & 1 & 1 \\
HvyAlcoholConsump & 0 & 0 & 0 & 0 \\
AnyHealthcare & 1 & 1 & 1 & 1 \\
NoDocbcCost & 0 & 0 & 1 & 0 \\
GenHlth & 5 & 4 & 4 & 5 \\
MentHlth & 26 & 18 & 21 & 20 \\
PhysHlth & \textbf{30} & \textbf{26} & \textbf{21} & \textbf{20} \\
DiffWalk & 1 & 1 & 1 & 1 \\
Sex & 0 & 1 & 1 & 0 \\
Age & 12 & 10 & 9 & 8 \\
Education & 5 & 6 & 3 & 5 \\
Income & 6 & 3 & 7 & 2 \\
\hline
\end{tabular}
}
\end{table}

High PhysHlth values are often accompanied by mobility limitations and low levels of physical activity, as observed through the \textit{PhysActivity} and \textit{DiffWalk} variables in the representative case study. Reduced physical activity typically reflects the co-occurrence of chronic conditions such as overweight and obesity, hypertension, dyslipidemia, and cardiovascular disease. These conditions contribute to insulin resistance and impaired glucose metabolism, thereby increasing the risk of diabetes.

Prior research has demonstrated that regular physical activity improves glycemic control and can prevent or delay the onset of type 2 diabetes, highlighting the critical role of physical functioning in metabolic health.

(Cite: https://pmc.ncbi.nlm.nih.gov/articles/PMC2992225/)

Therefore, very high PhysHlth values play a critical role in evaluating the relationship between physical health impairment and diabetes risk.

\begin{figure}[H]
\centering
\includegraphics[width=0.7\textwidth]{img/PhysHlth_category.png}
\caption{Distribution of PhysHlth categories}
\label{fig:physhlth_categories}
\end{figure}

Hình~\ref{fig:physhlth_categories} cho thấy một số lượng cá nhân báo cáo số ngày sức khỏe thể chất kém ở mức cao. Các trường hợp PhysHlth cao đại diện cho nhóm nguy cơ cao nhất và chứa thông tin quan trọng về cơ chế bệnh sinh cũng như dự báo bệnh lý.

\textbf{Conclusion:} \\
These findings support retaining PhysHlth outliers as epidemiologically meaningful observations rather than statistical artifacts, ensuring robust inference on the association between physical health impairment and diabetes risk.

\subsubsection{Mental Health (MentHlth)}

\begin{figure}[H]
\centering
\includegraphics[width=1\textwidth]{img/MentHlth_density.png}
\caption{Density plot of Mental Health distribution}
\label{fig:ment_density}
\end{figure}

\begin{figure}[H]
\centering
\includegraphics[width=1\textwidth]{img/MentHlth_boxplot.png}
\caption{Boxplot of Mental Health}
\label{fig:ment_boxplot}
\end{figure}

The boxplot and density plot of MentHlth show a strongly right-skewed distribution with many high-end extreme values. The boxplot indicates several observations beyond the upper whisker, while the density plot exhibits a sharp peak near zero and a long right tail. This suggests that most respondents reported few poor mental health days, but a small group experienced extremely high values, which may significantly affect statistical analysis.

\textbf{Outlier Case Study:} \\
After using density plots and IQR-based boxplots to identify outliers in the \textit{MentHlth} variable, the study further examined the epidemiological relevance of these extreme observations. To improve interpretability and reduce data sparsity at higher values, \textit{MentHlth} was categorized into four ordinal levels (0 days, 1--7 days, 8--14 days, and 15--30 days), representing increasing levels of mental health impairment over the past 30 days.

To assess the impact of extreme values, several representative observations in the 15–30 day category were selected for in-depth analysis. Detailed information on these cases is presented in Table~\ref{tab:menthlth_cases}.

\begin{table}[H]
\centering
\caption{Selected individuals with extreme mental health values (MentHlth outliers)}
\label{tab:menthlth_cases}

\scriptsize
\setlength{\tabcolsep}{4pt}
\renewcommand{\arraystretch}{1.1}

\begin{tabular}{l|c|c|c}
\hline
\textbf{Variable} & \textbf{Case 1 (MentHlth=30)} & \textbf{Case 2 (MentHlth=30)} & \textbf{Case 3 (MentHlth=28)} \\
\hline
Diabetes\_012 & 1 & 1 & 1 \\
HighBP & 1 & 1 & 1 \\
HighChol & 1 & 1 & 1 \\
CholCheck & 1 & 1 & 1 \\
BMI & 27 & 36 & 28 \\
Smoker & 0 & 1 & 1 \\
Stroke & 0 & 0 & 0 \\
HeartDiseaseorAttack & 0 & 0 & 1 \\
PhysActivity & 1 & 0 & 0 \\
Fruits & 1 & 0 & 1 \\
Veggies & 1 & 0 & 1 \\
HvyAlcoholConsump & 0 & 0 & 0 \\
AnyHealthcare & 1 & 1 & 1 \\
NoDocbcCost & 1 & 0 & 0 \\
GenHlth & 4 & 5 & 5 \\
MentHlth & 30 & 30 & 28 \\
PhysHlth & 30 & 21 & 24 \\
DiffWalk & 1 & 1 & 1 \\
Sex & 0 & 1 & 0 \\
Age & 10 & 9 & 9 \\
Education & 4 & 3 & 4 \\
Income & 3 & 5 & 2 \\
\hline
\end{tabular}
\end{table}


High MentHlth values often reflect prolonged psychological stress, anxiety, or depression, which may substantially influence health behaviors and metabolic-related physiological mechanisms. In the representative case study, the individual with MentHlth = 30 also reported mobility limitations (DiffWalk = 1) and relatively low levels of physical activity, suggesting a potential association between impaired mental health and a sedentary lifestyle. Chronic psychological disorders may lead to unhealthy behaviors such as poor dietary habits, reduced physical activity, and low treatment adherence, thereby exacerbating metabolic risk factors.

Moreover, prior literature has demonstrated a bidirectional relationship between diabetes and mental health disorders, with depression increasing the risk of diabetes and diabetes-related burden worsening mental health outcomes 

(Cite: https://www.ncbi.nlm.nih.gov/books/NBK498652/).

Therefore, very high MentHlth values play an important role in evaluating the relationship between impaired mental health and diabetes risk.

\begin{figure}[H]
\centering
\includegraphics[width=0.7\textwidth]{img/MentHlth_category.png}
\caption{Distribution of MentHlth categories}
\label{fig:menthlth_categories}
\end{figure}

Figure~\ref{fig:menthlth_categories} shows that a small proportion of individuals reported a high number of poor mental health days. High MentHlth cases represent a high-risk group and provide important information on psychosocial factors as well as the potential for disease risk prediction.

\textbf{Conclusion:} \\
Based on visual evidence and representative case analysis, this study decided to retain MentHlth outliers to ensure data representativeness and maintain the reliability of conclusions regarding the role of mental health status in diabetes risk.

\subsection{Summary of Preprocessing Decisions}

\begin{itemize}
    \item \textbf{Duplicate data assessment:} The duplicate data assessment revealed a moderate proportion of duplicated observations. These records were retained because, in large-scale survey data, identical records may represent different individuals with similar demographic and health characteristics. Removing such observations could distort the population distribution and reduce the statistical power of subsequent analyses.
    
    \item \textbf{Missing data assessment:} The missing data inspection indicated that no missing values were detected across all variables. Therefore, no missing data handling or imputation techniques were required.
    
    \item \textbf{Outlier assessment:} Outliers were identified using density plots and IQR-based boxplots for BMI, PhysHlth, and MentHlth. Extreme values were retained because they are epidemiologically meaningful and represent high-risk health profiles related to diabetes.
\end{itemize}
