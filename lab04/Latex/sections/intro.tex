\section{Introduction}
\subsection{Background and Motivation}

Diabetes is one of the most serious and increasingly prevalent public health problems worldwide. In the United States, the prevalence of diabetes has risen rapidly over recent decades, imposing a substantial burden on healthcare systems and population quality of life. In addition to biological factors, numerous studies have shown that lifestyle behaviors, demographic characteristics, and health indicators are closely associated with diabetes risk. However, the magnitude and nature of these relationships remain insufficiently understood and require further investigation through large-scale empirical data analysis.

\subsection{Data Description and Context}

This study utilizes data extracted from the Behavioral Risk Factor Surveillance System (BRFSS) 2015, a large-scale public health surveillance system in the United States. The data were collected through telephone surveys of adults aged 18 years and older, covering information on lifestyle behaviors, health status, demographic characteristics, and underlying medical conditions. With a large sample size and a wide range of variables, the BRFSS dataset provides an appropriate foundation for analyzing factors associated with diabetes at the population level.

\subsection{Research Questions}

To achieve the objectives of this study, the analysis is structured around the following research questions:

\subsubsection{Demographic Differences in Diabetes}
\begin{itemize}
    \item How does diabetes prevalence differ across demographic groups?
    \item How does age influence the likelihood of having diabetes?
    \item Is there a significant association between gender and diabetes status?
    \item How is access to healthcare related to diabetes prevalence?
\end{itemize}

\subsubsection{Lifestyle Factors and Diabetes Risk}
\begin{itemize}
    \item Do unhealthy lifestyle habits increase the risk of diabetes?
    \item How are healthy lifestyle behaviors associated with diabetes outcomes?
\end{itemize}

\subsubsection{Health Risk Factors and Predictive Power}
\begin{itemize}
    \item Does obesity status differ significantly between individuals with and without diabetes?
    \item Is there a statistically significant difference in cholesterol levels between diabetic and non-diabetic groups?
    \item Which health indicators are the most influential predictors of diabetes risk when applying machine learning models?
\end{itemize}





