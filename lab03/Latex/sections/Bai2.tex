\begin{question}{2}
    Trong một thống kê kiểm định so sánh hai trung bình, với giả thuyết gốc
    $H_0: \mu_1 \le \mu_2 \qquad$ và đối thuyết $\qquad H_1: \mu_1 > \mu_2$

    \begin{enumerate}
    \item a. Giá trị p-value nào sau đây, $0.03$ và $0.006$ là cung cấp bằng chứng mạnh mẽ hơn cho đối thuyết
$H_1$?
    \item b. Giá trị p-value nào sau đây, $0.095$ và $0.04$ là cung cấp bằng chứng mạnh mẽ hơn rằng chỉ có sự
ngẫu nhiên mới có thể giải thích cho kết quả quan sát được?
    \end{enumerate}

 %---- a) ----
    \begin{subquestion}{a.Giá trị p-value nào sau đây, $0.03$ và $0.006$ là cung cấp bằng chứng mạnh mẽ hơn cho đối thuyết
$H_1$?}
        \begin{answer}
        Theo định nghĩa, p-value được cho bởi: $p-value = \Pr_{H_0}(T \ge T_{\text{obs}})$, nên giá trị p-value càng nhỏ thì bằng chứng chống lại $H_0$ càng mạnh.
        
        Xét hai p-value: $0.03$ và $0.006$.
        Vì $0.006 < 0.03$
        Nên theo quy ước thống kê, p-value $0.006$ cung cấp bằng chứng mạnh hơn chống lại $H_0$ và ủng hộ đối thuyết $H_1 : \mu_1 > \mu_2$.

        Vậy: p-value = $0.006$ cung cấp bằng chứng mạnh mẽ hơn cho $H_1$.

        \end{answer}
    \end{subquestion}

%---- 6.b ----
       \begin{subquestion}{b. Giá trị p-value nào sau đây, $0.095$ và $0.04$ là cung cấp bằng chứng mạnh mẽ hơn rằng chỉ có sự
ngẫu nhiên mới có thể giải thích cho kết quả quan sát được?}
        \begin{answer}
        Theo định nghĩa, p-value được cho bởi: $p-value = \Pr_{H_0}(T \ge T_{\text{obs}})$, nên p-value lớn $\;\Longrightarrow\;$ Kết quả quan sát có khả năng cao xuất hiện do ngẫu nhiên.
            
        Ta xét hai giá trị: $0.095$ và $0.04$.
        Vì p-value càng lớn thì dữ liệu càng phù hợp với $H_0$, và càng có nhiều khả năng rằng kết quả quan sát chỉ do biến thiên ngẫu nhiên.
        Ta thấy: $0.095 > 0.04$
        Do đó p-value $0.095$ hàm ý rằng kết quả quan sát được dễ dàng xảy ra dưới $H_0$ và không đủ mạnh để bác bỏ giả thuyết.

        Vậy: p-value = $0.095$ cho thấy bằng chứng mạnh hơn rằng kết quả có thể do ngẫu nhiên.

        \end{answer}
        \end{subquestion}

\end{question}