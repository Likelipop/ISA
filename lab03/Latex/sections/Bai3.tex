\begin{question}{2}
    Trong một trò chơi dự đoán phân phối màu sắc của bi chứa trong một chiếc bình lớn, người quản trò nói rằng 
    bên trong có $9999$ viên bi xanhvà $1$ viên bi đỏ. Bạn (người chơi) rút ngẫu nhiên một viên bi và thấy nó màu đỏ.

    \begin{enumerate}
    \item a. Liệu bạn có sẵn lòng nói với người quản trò rằng bạn nghĩ người ấy đã sai về sự phân bố màu sắc không?
    Tại sao có hoặc tại sao không?
    \item b. Bạn đang giả định điều gì khi đưa ra quyết định của mình?
    \item c. Còn nếu thay vào đó, người quản trò nói rằng có $9$ viên bi xanh và $1$ viên bi đỏ (và bạn rút ra một viên bi đỏ) thì sao?
    \end{enumerate}

 %---- a) ----
    \begin{subquestion}{a.Liệu bạn có sẵn lòng nói với người quản trò rằng bạn nghĩ người ấy đã sai về sự phân bố màu sắc không?
    Tại sao có hoặc tại sao không?}
        \begin{answer}
        Theo định nghĩa, p-value được cho bởi: $p-value = \Pr_{H_0}(T \ge T_{\text{obs}})$, nên giá trị p-value càng nhỏ thì bằng chứng chống lại $H_0$ càng mạnh.
        
        Xét hai p-value: $0.03$ và $0.006$.
        Vì $0.006 < 0.03$
        Nên theo quy ước thống kê, p-value $0.006$ cung cấp bằng chứng mạnh hơn chống lại $H_0$ và ủng hộ đối thuyết $H_1 : \mu_1 > \mu_2$.

        Vậy: p-value = $0.006$ cung cấp bằng chứng mạnh mẽ hơn cho $H_1$.

        \end{answer}
    \end{subquestion}

%---- 6.b ----
       \begin{subquestion}{b. Giá trị p-value nào sau đây, $0.095$ và $0.04$ là cung cấp bằng chứng mạnh mẽ hơn rằng chỉ có sự
ngẫu nhiên mới có thể giải thích cho kết quả quan sát được?}
        \begin{answer}
        Theo định nghĩa, p-value được cho bởi: $p-value = \Pr_{H_0}(T \ge T_{\text{obs}})$, nên p-value lớn $\;\Longrightarrow\;$ Kết quả quan sát có khả năng cao xuất hiện do ngẫu nhiên.
            
        Ta xét hai giá trị: $0.095$ và $0.04$.
        Vì p-value càng lớn thì dữ liệu càng phù hợp với $H_0$, và càng có nhiều khả năng rằng kết quả quan sát chỉ do biến thiên ngẫu nhiên.
        Ta thấy: $0.095 > 0.04$
        Do đó p-value $0.095$ hàm ý rằng kết quả quan sát được dễ dàng xảy ra dưới $H_0$ và không đủ mạnh để bác bỏ giả thuyết.

        Vậy: p-value = $0.095$ cho thấy bằng chứng mạnh hơn rằng kết quả có thể do ngẫu nhiên.

        \end{answer}
        \end{subquestion}

\end{question}