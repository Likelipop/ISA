\begin{question}{3}
<<<<<<< HEAD
    Trong một trò chơi dự đoán phân phối màu sắc của bi chứa trong một chiếc bình lớn, người quản trò nói rằng 
    bên trong có $9999$ viên bi xanh và $1$ viên bi đỏ. Bạn (người chơi) rút ngẫu nhiên một viên bi và thấy nó màu đỏ.

    \begin{enumerate} [label=\alph*)]
    \item Liệu bạn có sẵn lòng nói với người quản trò rằng bạn nghĩ người ấy đã sai về sự phân bố màu sắc không?
    Tại sao có hoặc tại sao không?
    \item Bạn đang giả định điều gì khi đưa ra quyết định của mình?
    \item Còn nếu thay vào đó, người quản trò nói rằng có $9$ viên bi xanh và $1$ viên bi đỏ (và bạn rút ra một viên bi đỏ) thì sao?
    \end{enumerate}

 %---- a) ----
    \begin{subquestion}{Liệu bạn có sẵn lòng nói với người quản trò rằng bạn nghĩ người ấy đã sai về sự phân bố màu sắc không?
    Tại sao có hoặc tại sao không?}
        \begin{answer}
        Giả thuyết gốc ($H_0$) là $P(\text{Đỏ}) = 1/10000$.
        $$p\text{-value} = P(\text{Đỏ} \mid H_0) = \frac{1}{9999 + 1} = \mathbf{0.0001}$$

        Kết luận: bạn nên thách thức giả định này.

        Lý do: Giá trị $p\text{-value} = 0.0001$ là cực kỳ nhỏ, thấp hơn rất nhiều so với mọi mức ý nghĩa tiêu chuẩn
        ($\alpha = 0.05$). Việc quan sát được bi Đỏ là một sự kiện vô cùng hiếm hoi nếu $H_0$ là đúng. Theo nguyên tắc thống kê,
        một sự kiện hiếm hoi xảy ra cung cấp bằng chứng mạnh mẽ chống lại giả thuyết gốc. Chúng ta bác bỏ $H_0$.

        \end{answer}
    \end{subquestion}

%---- b) ----
       \begin{subquestion}{Bạn đang giả định điều gì khi đưa ra quyết định của mình?}
        \begin{answer}
        Khi đưa ra quyết định thách thức $H_0$ dựa trên xác suất, người chơi đang dựa trên giả định nền tảng của mọi thí nghiệm thống kê:

        Giả định Nền tảng: Việc rút viên bi là một mẫu ngẫu nhiên độc lập ($i.i.d.$ observation) từ quần thể.

        Quá trình rút bi là Truly Random và mọi viên bi có cơ hội được rút ra như nhau.

        Giải thích: Nếu quá trình rút bi bị thiên vị (không ngẫu nhiên), thì kết luận bác bỏ $H_0$ 
        của bạn sẽ bị sai lệch, vì lỗi nằm ở thiết kế thí nghiệm chứ không phải lỗi ở phân bố quần thể.

        \end{answer}
        \end{subquestion}

%---- c) ----
       \begin{subquestion}{Còn nếu thay vào đó, người quản trò nói rằng có $9$ viên bi xanh và $1$ viên bi đỏ (và bạn rút ra một viên bi đỏ) thì sao?}
        \begin{answer}
        Giả thuyết gốc mới là $H_0': P(\text{Đỏ}) = 1/10$.
        $$p\text{-value}' = P(\text{Đỏ} \mid H_0') = \frac{1}{10} = \mathbf{0.1}$$

        Kết luận: Bạn không nên thách thức người quản trò.

        Lý do: Giá trị $p\text{-value}' = 0.1$ là lớn hơn mức ý nghĩa tiêu chuẩn $\alpha = 0.05$. 

        Vì $0.1 > 0.05$, chúng ta không bác bỏ giả thuyết gốc $H_0'$. Việc rút được bi đỏ là một kết quả hợp lý và có thể được giải thích là do sự ngẫu nhiên trong phân bố $9:1$. 

        Chúng ta chấp nhận rằng tuyên bố của người quản trò là có thể chấp nhận được (plausible).

        \end{answer}
        \end{subquestion}
=======
    Đặt $\bar{y}$ và $m$ lần lượt là trung bình và trung vị của mẫu $y_1 < y_2 < \dots < y_n$. Xét $f$ là một hàm số thực.

    \begin{enumerate}
    \item a. Có phải $f(\bar{y})$ là trung bình mẫu của dữ liệu $f(y_1), f(y_2), \dots, f(y_n)$?
    \item b. Có phải $f(m)$ là trung vị của dữ liệu $f(y_1), f(y_2), \dots, f(y_n)$?
    \item c. Có hay không bất kỳ điều kiện gì để chắc chắn rằng $f(\bar{y})$ là trung bình mẫu của dữ liệu đã biến đổi?
    \item d. Có hay không bất kỳ điều kiện gì để chắc chắn rằng $f(m)$ là trung vị của dữ liệu đã biến đổi?
    \end{enumerate}

    Gọi $Y = \{y_1, y_2, \dots, y_n\}$ là tập hợp dữ liệu gốc.
    Gọi $Z = \{f(y_1), f(y_2), \dots, f(y_n)\}$ là tập hợp dữ liệu đã qua phép biến đổi $f$.

 %---- 6.a ----
    \begin{subquestion}{a.Có phải $f(\bar{y})$ là trung bình mẫu của dữ liệu $f(y_1), f(y_2), \dots, f(y_n)$?}
        \begin{answer}
            Ta có: $\bar{y} = \frac{1}{n} \sum_{i=1}^{n} y_i$.
            Trung bình mẫu của dữ liệu đã biến đổi $Z$ là $\bar{z} = \frac{1}{n} \sum_{i=1}^{n} f(y_i)$.
            Câu hỏi yêu cầu chúng ta kiểm tra liệu: $f(\bar{y}) = \bar{z}$?
            Nói cách khác: $f\left( \frac{1}{n} \sum_{i=1}^{n} y_i \right) = \frac{1}{n} \sum_{i=1}^{n} f(y_i)$

            Chứng minh (Phản ví dụ):
            Chọn $f(x) = x^2$
            Giả sử: Mẫu $Y = \{1, 3\}$

            \begin{itemize}
            \item $\bar{y} = \frac{1+3}{2} = 2$
            \item $f(\bar{y}) = f(2) = 2^2 = 4$
            \item Xét $Z = \{f(1), f(3)\} = \{1^2, 3^2\} = \{1, 9\}$.
            \item Trung bình mẫu của $Z$: $\bar{z} = \frac{1+9}{2} = 5$
            \end{itemize}

            Vì $f(\bar{y}) = 4 \neq 5 = \bar{z}$.
            Nên ta có kết luận: (a): $f(\bar{y})$ không phải là trung bình mẫu của dữ liệu $f(y_1), f(y_2), \dots, f(y_n)$. 
        \end{answer}
    \end{subquestion}

%---- 6.b ----
       \begin{subquestion}{b. Có phải $f(m)$ là trung vị của dữ liệu $f(y_1), f(y_2), \dots, f(y_n)$?
                              hay $f(m) = m_Z$ luôn đúng với mọi hàm $f$ hay không}
        \begin{answer}
            Chứng minh (Phản ví dụ):
            
            TH1: Hàm không đơn điệu, n lẻ (n là kích cỡ mẫu $Y$ và $Z$)
            Chọn $f(x) = (x-5)^2$
            Giả sử: mẫu $Y = \{1, 2, 10\}$ và $f(m) = m_Z$ là đúng
            Khi đó: Mẫu đã được sắp xếp $y_1=1, y_2=2, y_3=10$.
            \begin{itemize}
                \item $m = y_2 = 2$ 
                \item $f(m) = f(2) = (2-5)^2 = 9$
                \item Xét $Z = \{f(1), f(2), f(10)\}$:
                \begin{itemize}
                    \item $f(1) = (1-5)^2 = 16$
                    \item $f(2) = (2-5)^2 = 9$
                    \item $f(10) = (10-5)^2 = 25$
                \end{itemize}
                \item Mẫu $Z = \{16, 9, 25\}$. Chúng ta phải sắp xếp lại mẫu này: $Z_{(ordered)} = \{9, 16, 25\}$.
                \item Trung vị của $Z$: $m_Z = 16$  
            \end{itemize}
            Vì $f(m) = 9 \neq 16 = m_Z$
            Nên $f(m) \neq m_Z$ trong hàm không đơn điệu

            TH2: Hàm đơn điệu, n chẵn (n là kích cỡ mẫu $Y$ và $Z$)
            Chọn $f(x) = x^3$ và $f(m) = m_Z$ là đúng
            Giả sử: $Y = \{1, 3\}$
            \begin{itemize}
                \item $m = \frac{1+3}{2} = 2$ 
                \item $f(m) = f(2) = 2^3 = 8$
                \item Xét $Z = \{f(1), f(3)\} = \{1^3, 3^3\} = \{1, 27\}$. Vì $f$ đơn điệu tăng, $Z$ đã được sắp xếp.
                \item Trung vị của $Z$: $m_Z = \frac{1+27}{2} = 14$
            \end{itemize}
            Vì $f(m) = 8 \neq 14 = m_Z$
            Nên $f(m) \neq m_Z$ trong hàm đơn điệu

        Từ TH1 và TH2, ta có kết luận (b): $f(m)$ không phải là trung vị của dữ liệu $f(y_1), f(y_2), \dots, f(y_n)$.
        \end{answer}
    \end{subquestion}

%---- 6.c ----
    \begin{subquestion}{c. Có hay không bất kỳ điều kiện gì để chắc chắn rằng $f(\bar{y})$ là trung bình mẫu 
                    ủa dữ liệu đã biến đổi?}
        \begin{answer}
            Có. Chúng ta cần tìm điều kiện để $f(\bar{y}) = \bar{z}$, tức là:
            $$f\left( \frac{1}{n} \sum_{i=1}^{n} y_i \right) = \frac{1}{n} \sum_{i=1}^{n} f(y_i)$$
 
            Theo Bất đẳng thức Jensen, đẳng thức xảy ra khi và chỉ khi $f$ là một hàm affine.
            Hay $f(x) = a\cdot x + b$ với $a, b \in \mathbb{R}$ là các hằng số.

            Ta sẽ chứng minh điều kiện này:
            \begin{itemize}
                \item Vế trái (LHS):
                \begin{align*} 
                    f(\bar{y}) &= f\left( \frac{1}{n} \sum_{i=1}^{n} y_i \right) \\ 
                               &= a \left( \frac{1}{n} \sum_{i=1}^{n} y_i \right) + b 
                \end{align*}
                \item Vế phải (RHS):
                \begin{align*} 
                    \bar{z} &= \frac{1}{n} \sum_{i=1}^{n} f(y_i) \\
                     &= \frac{1}{n} \sum_{i=1}^{n} (ay_i + b) \\ 
                     &= \frac{1}{n} \left( \sum_{i=1}^{n} (ay_i) + \sum_{i=1}^{n} b \right) \\ 
                     &= \frac{1}{n} \left( a \sum_{i=1}^{n} y_i + nb \right) \\  
                     &= a \left( \frac{1}{n} \sum_{i=1}^{n} y_i \right) + b 
                \end{align*}
            \end{itemize}
            Vì LHS = RHS, nên điều kiện được thỏa mãn.

         Kết luận (c): Có, điều kiện là $f$ phải là một hàm affine.
        \end{answer}
    \end{subquestion}

%---- 6.d ----
    \begin{subquestion}{d. Có hay không bất kỳ điều kiện gì để chắc chắn rằng $f(m)$ là trung vị của dữ liệu đã biến đổi?}
        \begin{answer}
            Chứng minh:
            
            TH1:Trường hợp tổng quát, đúng cho mọi $n$ ($n là kích thước mẫu$)
            Điều kiện để $f(m) = m_Z$ đúng cho mọi $n$ (cả chẵn và lẻ) là $f$ phải là một hàm affine ($f(x) = ax+b$).

            Khi đó, ta có: $f(m) = f\left(\frac{y_k + y_{k+1}}{2}\right) = a\left(\frac{y_k + y_{k+1}}{2}\right) + b$.
            \begin{itemize}
                \item Nếu $a \ge 0$, $f$ không giảm, mẫu $Z$ được sắp xếp là $f(y_1), \dots, f(y_n)$.
                $m_Z = \frac{f(y_k) + f(y_{k+1})}{2} = \frac{(ay_k+b) + (ay_{k+1}+b)}{2} = a\left(\frac{y_k + y_{k+1}}{2}\right) + b$.
                \item Nếu $a < 0$, $f$ không tăng, mẫu $Z$ được sắp xếp ngược lại: $f(y_n) \le \dots \le f(y_1)$. Các phần tử giữa là $f(y_{k+1})$ và $f(y_k)$.
                $m_Z = \frac{f(y_{k+1}) + f(y_k)}{2} = \frac{(ay_{k+1}+b) + (ay_k+b)}{2} = a\left(\frac{y_k + y_{k+1}}{2}\right) + b$.
            \end{itemize}
            Trong mọi trường hợp, $f(m) = m_Z$ khi $f$ là affine.

            TH2: Trường hợp đặc biệt, $n$ lẻ ($n là kích thước mẫu$)
            Nếu chúng ta được đảm bảo rằng kích thước mẫu $n$ là lẻ ($n=2k+1$), thì một điều kiện yếu hơn (ít nghiêm ngặt hơn) là đủ:
            $f$ chỉ cần là một hàm đơn điệu (monotonic).
            
            Nếu $n=2k+1$, thì $m = y_{k+1}$. Do đó $f(m) = f(y_{k+1})$.
            \begin{itemize}
            \item Trường hợp 1: $f$ đơn điệu không giảm.
            Vì $y_1 \le \dots \le y_n$, chúng ta có $f(y_1) \le \dots \le f(y_n)$.
            Mẫu $Z$ đã được sắp xếp. Trung vị $m_Z$ là phần tử thứ $(k+1)$ của $Z$, tức là $m_Z = f(y_{k+1})$.
            Vậy $f(m) = m_Z$.

            \item Trường hợp 2: $f$ đơn điệu không tăng.
            Vì $y_1 \le \dots \le y_n$, chúng ta có $f(y_1) \ge \dots \ge f(y_n)$.
            Mẫu $Z$ được sắp xếp theo thứ tự ngược lại: $f(y_n) \le \dots \le f(y_{k+1}) \le \dots \le f(y_1)$.
            Trung vị $m_Z$ vẫn là phần tử chính giữa (thứ $k+1$ trong dãy đã sắp xếp), tức là $m_Z = f(y_{k+1})$.
            Vậy $f(m) = m_Z$.
            \end{itemize}
        
        Từ TH1 và TH2: Kết luận (d): Có điều kiện để chắc chắn rằng $f(m)$ là trung vị của dữ liệu đã biến đổi.
        \end{answer}
    \end{subquestion}

>>>>>>> a14343c (Khởi tạo lại repo sạch)
\end{question}