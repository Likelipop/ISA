\begin{question}{8}
Xét bài toán phân tích dữ liệu về lượng asen thải ra trong nước thải 
từ một nhà máy công nghiệp. 
Muc tiêu là kiểm tra xem lượng asen thải ra 
có vượt quá giá trị trung bình 10 mg/lít
theo công bố của công ty hay không. 
Để đưa ra kết luận trong kiểm định một phía, chúng ta sử
dụng $\alpha$ = 0.05. Trong các nhận xét sau, nhận xét nào là đúng?

% \begin{enumerate}[label=(\alph*)]
%     \item Nếu $\mu = 10$, với xác suất $0.05$, 
%     chúng ta sẽ kết luận rằng lượng asen thải ra 
%     là vượt quá giới hạn.

%     \item Nếu $\mu = 10$, xác suất mà trung bình mẫu 
%     ($\overline{Y}$) bằng giá trị 
%     trung bình mẫu của dữ liệu quan sát là $0.05$.

%     \item Nếu lượng asen thải ra của nhà máy là 
%     vượt quá giới hạn, xác suất mà chúng ta 
%     sẽ kết luận rằng lượng asen thải ra 
%     không vượt quá giới hạn là $0.05$.

%     \item Nếu chúng ta bác bỏ giả thuyết $H_0$, 
%     xác suất mà $H_0$ thực sự đúng là $0.05$.

% \end{enumerate}

\begin{subquestion}{Nếu $\mu = 10$, với xác suất $0.05$, 
    chúng ta sẽ kết luận rằng lượng asen thải ra 
    là vượt quá giới hạn.}
    \begin{answer}

    \textbf{Câu này là chính xác.}

    Rõ ràng, đây là định nghĩa về sai lầm loại I,
    cụ thể chính là mức ý nghĩa $\alpha$:

    \begin{align*}
        \alpha = P(\text{bác bỏ } H_0 | H_0 \text{ đúng}) = 0.05
    \end{align*}

    \end{answer}
\end{subquestion}

\newpage
\begin{subquestion}{Nếu $\mu = 10$, xác suất mà trung bình mẫu 
    ($\overline{Y}$) bằng giá trị 
    trung bình mẫu của dữ liệu quan sát là $0.05$.}
    \begin{answer}
    Bằng biểu diễn toán học, ta sẽ mô tả câu này như sau:
    \begin{align*}
        P(\overline{Y} = \overline{y} | \mu = 10) = 0.05
    \end{align*}

    Ở đây, $\overline{y}$ là giá trị trung bình mẫu
    được tính từ dữ liệu quan sát. Và hiển nhiên, ta rút 
    ra một số nhận xét như sau: 

    \begin{itemize}
        \item Thứ nhất, do $\overline{Y}$ là biến ngẫu nhiên liên tục,
    nên xác suất để nó nhận một giá trị cụ thể
    (trong trường hợp này là $\overline{y}$)
    sẽ luôn bằng 0. Do đó, câu này không đúng.

        \item Thứ hai, ngay cả khi ta sửa lại câu này
    thành xác suất để $\overline{Y}$ nằm trong một khoảng nhỏ
    xung quanh $\overline{y}$, thì câu này vẫn không đúng.
    Bởi vì, xác suất này phụ thuộc vào nhiều yếu tố khác nhau,
    bao gồm phân phối của dữ liệu, kích thước mẫu,
    và độ lệch chuẩn của quần thể.
    Nó không thể được xác định chỉ 
    dựa trên giả định rằng $\mu = 10$.

    \end{itemize}
    

    

    
    \textbf{Câu này là sai.}
    \end{answer}
\end{subquestion}

\begin{subquestion}{Nếu lượng asen thải ra của nhà máy là 
    vượt quá giới hạn, xác suất mà chúng ta 
    sẽ kết luận rằng lượng asen thải ra 
    không vượt quá giới hạn là $0.05$.}

    \begin{answer}
        Ta sẽ mô tả lại câu này bằng công thức toán học như sau:
        \begin{align*}
            P(\text{không bác bỏ } H_0 | \mu > 10) = 0.05
        \end{align*}

        Ở đây, $\mu > 10$ biểu thị rằng lượng asen thải ra
        của nhà máy vượt quá giới hạn cho phép.
        Câu này đề cập đến khái niệm sai lầm loại II, tức 
        là xác suất không bác bỏ giả thuyết gốc
        khi giả thuyết đối là đúng. Hay được định nghĩa 
        trong môn lý thuyết thống kê là $\beta$

    
    \textbf{Câu này là sai.}
    \end{answer}
\end{subquestion}

\begin{subquestion}{Nếu chúng ta bác bỏ giả thuyết $H_0$, 
    xác suất mà $H_0$ thực sự đúng là $0.05$.}

    \begin{answer}
        Để ý, câu này ngược với câu (a) phía trên :
        \begin{align*}
            P(H_0 \text{ đúng } | \text{bác bỏ } H_0) = 0.05
        \end{align*}

        Câu này không đúng vì mức ý nghĩa $\alpha$
        không phải được định nghĩa như thế này. 
        Mà ngay cả khi sử dụng định nghĩa về
        mức ý nghĩa $\alpha$ để suy luận ngược lại,
        ta cũng không thể kết luận được điều này. Cụ thể
        \begin{align*}
            \alpha &= P(\text{bác bỏ } H_0 | H_0 \text{ đúng}) \\
            &\neq P(H_0 \text{ đúng } | \text{bác bỏ } H_0)
        \end{align*}

    \textbf{Câu này là sai.}
    \end{answer}
\end{subquestion}





\end{question}