\begin{question}{4}
<<<<<<< HEAD
Một công ty quản lý phân phối sản phẩm tới các cửa hàng bán lẻ, đã thực hiện thu
thập dữ liệu về giá niêm yết của cùng 1 sản phẩm từ hai cửa hàng. 
Có tất cả n sản phẩm được
quan sát. Mục tiêu của công ty là so sánh giá niêm yết của cùng một sản phẩm tại hai cửa hàng
khác nhau nhằm phát hiện gian lận thương mại.


 %---- 6.a ----
    \begin{subquestion}{Dựa vào cách thiết kế và thu thập dữ liệu, 
        hãy giải thích tại sao đây là ví dụ về dữ liệu ghép cặp.}
        \begin{answer}
            Dựa vào cách thiết kế thu thập dữ liệu được mô tả phía trên,
            ta đã tiến hành lấy mẫu ngẫu nhiên n sản phẩm từ cả hai cửa hàng.
            Ta đặt :
            \begin{align*}
                Y_{1i} &: \text{giá niêm yết của sản phẩm thứ i tại cửa hàng 1} \\
                Y_{2i} &: \text{giá niêm yết của sản phẩm thứ i tại cửa hàng 2} \\
            \end{align*}

            Như vậy, ta có thể thấy rằng giá niêm yết của cùng một sản phẩm tại hai cửa hàng
            khác nhau sẽ có sự liên kết với nhau, tức là giá niêm yết của sản phẩm 
            thứ i tại cửa hàng 1
            sẽ liên quan đến giá niêm yết của sản phẩm thứ i tại cửa hàng 2.
            Do đó, ta có thể kết luận rằng đây là ví dụ về dữ liệu ghép cặp.
        \end{answer}

    \end{subquestion}

 %---- 6.b ----
    \begin{subquestion}{Hãy phát biểu giả thuyết gốc H0 
        và đối thuyết H1 theo mục tiêu của công ty, tương ứng với
thiết kế thu thập dữ liệu.}
    \begin{answer}
        Mục tiêu của công ty là so sánh giá niêm yết của cùng một sản phẩm tại hai cửa hàng
        khác nhau nhằm phát hiện gian lận thương mại. Do đó, ta có thể phát biểu giả thuyết gốc
        và đối thuyết như sau:
        \begin{align*}
            H_0 &: \mu_D = 0 \quad \text{(không có sự khác biệt về giá niêm yết giữa hai cửa hàng)} \\
            H_1 &: \mu_D \neq 0 \quad \text{(có sự khác biệt về giá niêm yết giữa hai cửa hàng)}
        \end{align*}
        Trong đó, $\mu_D$ là trung bình của các hiệu giá niêm yết giữa hai cửa hàng cho cùng một sản phẩm.
=======
Xét hai bộ dữ liệu $x_1 < x_2 < x_3< ... < x_n $ và $y_1 < y_2 < y_3< ... < y_n $, 
có trung bình mẫu tương ứng là $\bar{x}$, $\bar{y}$ và trung
vị lần lượt là $m_x$ và $m_y$. Đặt $w_i = x_i + y_i$. 


 %---- 6.a ----
    \begin{subquestion}{Chứng minh hoặc đưa ra phản chứng rằng:
        $\bar{x} + \bar{y}$ là trung bình mẫu của $w_1, w_2,...,w_n$}
        \begin{answer}
            Trước hết, ta xét trung bình mẫu của $w_1, w_2,...,w_n$ :

        \begin{align*}
        \bar{w} 
            &= \frac{1}{n} \sum_{i=1}^{n} w_i  \\
            &= \frac{1}{n} \sum_{i=1}^{n} (x_i + y_i) \\
            &= \frac{1}{n} \left( \sum_{i=1}^{n} x_i + \sum_{i=1}^{n} y_i \right) \\
            &= \frac{1}{n} \sum_{i=1}^{n} x_i + \frac{1}{n} \sum_{i=1}^{n} y_i \\
            &= \bar{x} + \bar{y}
        \end{align*}

        Vậy suy ra $\bar{w}  = \bar{x} + \bar{y}$ là trung bình mẫu của $w_1, w_2,...,w_n$.
        \end{answer}
    \end{subquestion}

 %---- 6.b ----
    \begin{subquestion}{Chứng minh hoặc đưa ra phản chứng rằng:
        $m_x + m_y$ là trung vị của $w_1, w_2,\ldots,w_n$}
    \begin{answer}
        Ta sẽ chứng minh mệnh đề trên là đúng, trước hết ta sẽ xét với trường hợp n là số lẻ,
        tức rằng ta có trung vị của $\{x\}_n$ và $\{y\}_n$  lần lượt là:
        \[ m_x = x_{\frac{n+1}{2}} \]
        \[ m_y = y_{\frac{n+1}{2}} \]

        Khi đó với dãy $w$ ta có:

        \begin{align}
        m_w
            &= w_{\frac{n+1}{2}}  \notag\\
            &= x_{\frac{n+1}{2}} + y_{\frac{n+1}{2}} \notag\\
            &= m_x + m_y \label{eq:odd}
        \end{align}

        Với trường hợp n là số chẵn, ta có trung vị của $\{x\}_n$ và $\{y\}_n$ lần lượt là:
        \[ m_x = \frac{1}{2}(x_{\frac{n}{2}} + x_{\frac{n}{2} + 1}) \]
        \[ m_y = \frac{1}{2}(y_{\frac{n}{2}} + y_{\frac{n}{2} + 1})  \]

        Khi đó với dãy $w$ ta có:

        \begin{align}
        m_w
            &= \frac{1}{2}(w_{\frac{n}{2}} + w_{\frac{n}{2} + 1})  \notag\\
            &= \frac{1}{2}((x_{\frac{n}{2}} + y_{\frac{n}{2}}) + (x_{\frac{n}{2} + 1} + y_{\frac{n}{2} + 1} )  ) \notag\\
            &= \frac{1}{2}(x_{\frac{n}{2}} + x_{\frac{n}{2} + 1}) + \frac{1}{2}(y_{\frac{n}{2}} + y_{\frac{n}{2} + 1}) \notag \\
            &= m_x + m_y \label{eq:even}
        \end{align}

        Từ \eqref{eq:even} và \eqref{eq:odd} ta suy ra  
        $m_x + m_y$ là trung vị của $w_1, w_2,\ldots,w_n$ với mọi n.
>>>>>>> a14343c (Khởi tạo lại repo sạch)

    \end{answer}
  \end{subquestion}

<<<<<<< HEAD
  \begin{subquestion}{Hãy xây dựng quy trình kiểm định hoán vị cho giả thuyết gốc H0 và đối thuyết H1 được phát
biểu trong ý (b) sao cho tương ứng với dữ liệu ghép cặp.}

    \begin{answer}
        Để xây dựng quy trình kiểm định hoán vị cho giả thuyết gốc H0 và đối thuyết H1,
         ta có thể thực hiện các bước sau:
        \begin{enumerate}
            \item Tính $t_{diff} = \bar{Y_1} - \bar{Y_2}$ từ dữ liệu quan sát ban đầu
            \item Gộp chung hai dữ liệu thành một dữ liệu chung, ký hiệu chung là $Y$.
            với cỡ mẫu là $2n$
            \item Hoán vị các vị trí trong dữ liệu chung, sau đó, chia ngẫu nhiên dữ liệu thành hai bộ dữ
liệu có kích cỡ bằng nhau là $n$.
            \item Tính toán hiệu trung bình sau hoán vị của hai nhóm:
            \[
                t_{diff,\pi} = bar{Y_{1,\pi}} - bar{Y_{2,\pi}}
            \]
            \item Lập lại bước 3 và 4 trong n! lần
            \item Tính P-value :
            \[
                p\text{-value} = \frac{\text{số lần } |t_{diff,\pi}| \geq |t_{diff}|}{\text{tổng số hoán vị}}
            \]
            \item Quyết định kết luận:
            \begin{itemize}
                \item Nếu p-value nhỏ hơn mức ý nghĩa $\alpha$ (ví dụ: 0.05), bác bỏ giả thuyết gốc H0.
                \item Ngược lại, không đủ bằng chứng để bác bỏ H0.
            \end{itemize}
        \end{enumerate}

    \end{answer}

\end{subquestion}

\begin{subquestion}{Hãy xây dựng quy trình kiểm định Bootstrap cho giả thuyết 
    gốc H0 và đối thuyết H1 tương ứng với dữ liệu ghép cặp.}

\begin{answer}
    Để xây dựng quy trình kiểm định Bootstrap, 
    ta sẽ tập trung vào hiệu số giữa các cặp quan sát và sử dụng phương pháp *
    *Bootstrap để kiểm định giả thuyết .
    \begin{enumerate}
        \item  Tính toán Thống kê Quan sát: 
        \begin{itemize}
            \item Tính  hiệu số  $D_i = Y_{1,i} - Y_{2,i}$ cho mỗi cặp dữ liệu $i = 1, \dots, n$.
            \item Tính toán  thống kê kiểm định quan sát  $T_{obs} = \bar{D}$ (trung bình của các hiệu số $D_i$).
        \end{itemize}
        \item  Dịch chuyển dữ liệu (Shifting): 
        \begin{itemize}
            \item Theo giả thuyết $H_0: \mu_d = 0$, ta cần tạo ra một tập dữ liệu  mẫu Bootstrap  tuân theo $H_0$.
            \item  Dịch chuyển  (center) tập hiệu số $D$ bằng cách tính $D^* = D - \bar{D}$. Khi đó, $\bar{D^*} = 0$, phù hợp với $H_0$.
        \end{itemize}
        \item  Lặp lại quá trình Bootstrap $B$ lần (ví dụ: $B=10,000$): 
        \begin{enumerate}
            \item  Lấy mẫu Bootstrap:  Rút mẫu ngẫu nhiên  có hoàn lại  $n$ hiệu số từ tập $D^*$ để tạo ra tập mẫu Bootstrap $D_b^* = \{D_{b,1}^*, \dots, D_{b,n}^*\}$.
            \item  Tính toán Thống kê Bootstrap:  Tính toán  thống kê kiểm định Bootstrap  $T_b$ từ mẫu $D_b^*$:
            \[
                T_b = \bar{D_b^*} = \frac{1}{n} \sum_{i=1}^{n} D_{b,i}^*
            \]
        \end{enumerate}
        \item  Xây dựng Phân phối Mẫu Bootstrap: 
        \begin{itemize}
            \item Tập hợp tất cả các giá trị $T_b$ thu được để tạo ra  phân phối mẫu Bootstrap  dưới $H_0$.
        \end{itemize}
        \item  Tính toán P-value: 
        \begin{itemize}
            \item Giá trị $p$ được tính bằng tỷ lệ các thống kê Bootstrap có giá trị bằng hoặc lớn hơn (theo trị tuyệt đối) thống kê quan sát $T_{obs}$ (đối với kiểm định hai phía):
            \[
                p\text{-value} = \frac{\text{số lần } |T_b| \geq |T_{obs}|}{B}
            \]
        \end{itemize}
        \item  Quyết định kết luận: 
        \begin{itemize}
            \item Nếu $p$-value nhỏ hơn mức ý nghĩa $\alpha$ (ví dụ: 0.05),  bác bỏ giả thuyết gốc $H_0$ .
            \item Ngược lại,  không đủ bằng chứng để bác bỏ $H_0$ .
        \end{itemize}
    \end{enumerate}

\end{answer}
\end{subquestion}

=======
>>>>>>> a14343c (Khởi tạo lại repo sạch)
\end{question}

