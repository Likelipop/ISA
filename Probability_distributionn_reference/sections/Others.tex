\section{Conclusion}
Đây là phần kết của tài liệu \cite{einstein1905}.
\section{Phân phối Gamma (Gamma Distribution)}
Phân phối Gamma là một phân phối xác suất liên tục, dương và có hình dạng linh hoạt, thường được sử dụng để mô hình hóa thời gian chờ đợi hoặc các đại lượng dương có tính chất kéo dài.
	
\begin{dinhnghia}[Hàm mật độ xác suất (Probability Density Function - PDF)]
Một biến ngẫu nhiên $X$ tuân theo phân phối Gamma với tham số hình dạng (shape parameter) $k > 0$ và tham số tỷ lệ (scale parameter) $\theta > 0$, ký hiệu $X \sim \text{Gamma}(k, \theta)$, có hàm mật độ xác suất được cho bởi:
\[ f(x; k, \theta) = \frac{x^{k-1} e^{-x/\theta}}{\Gamma(k) \theta^k} \quad \text{với } x > 0 \]
Trong đó, $\Gamma(k)$ là hàm Gamma, được định nghĩa là $\Gamma(k) = \int_0^\infty t^{k-1} e^{-t} dt$.
\end{dinhnghia}
	
