\section{Permutation Test}

\begin{frame}{Permutation test}
    \begin{frame}{Objectives}
    \begin{enumerate}
        \item \textbf{Analysis of Time-on-Page}
            \begin{itemize}
                \item \textbf{Research Question:} Is there a significant difference in time spent?
                \item \textbf{Null Hypothesis ($H_0$):} $\mu_{\text{new}} = \mu_{\text{old}}$
                \item \textbf{Alternative ($H_1$):} $\mu_{\text{new}} \neq \mu_{\text{old}}$ (Two-sided)
            \end{itemize}
        \vfill
        \item \textbf{Analysis of Conversion Rate}
            \begin{itemize}
                \item \textbf{Research Question:} Is there a significant difference in conversion rate?
                \item \textbf{Null Hypothesis ($H_0$):} $p_{\text{new}} = p_{\text{old}}$
                \item \textbf{Alternative ($H_1$):} $p_{\text{new}} \neq p_{\text{old}}$ (Two-sided)
            \end{itemize}
    \end{enumerate}
    \vfill
    \begin{block}{Methodology}
    Both analyses will utilize permutation sampling (two-sided test).
    \end{block}
\end{frame}

% =========================================
\section{Methodology}
% =========================================

\begin{frame}{Methodology: Generalized Permutation Test}
    The \texttt{perm\_test()} function encapsulates a 6-step logic (from the diagram):
    \begin{enumerate}
        \item \textbf{Step 1: Compute $t_{\text{obs}}$ (Observed Statistic):}
        Calculate the observed statistic from the original data.
        \[ t_{\text{obs}} = \bar{Y}_{\text{new}} - \bar{Y}_{\text{old}} \]
        \vfill
        \item \textbf{Step 2: Pool Data:}
        Combine all observations (Y1, ..., Yn) under the null hypothesis.
        \vfill
        \item \textbf{Steps 3 \& 5: Shuffle and Repeat:}
        Repeat R=10,000 times: randomly shuffle the \texttt{group} labels.
        \vfill
        \item \textbf{Step 4: Compute $t^\ast$ (Permuted Statistic):}
        Recalculate the statistic ($t^\ast$) for each permuted dataset.
        \[ t^\ast = \bar{Y}^\ast_{\text{new}} - \bar{Y}^\ast_{\text{old}} \]
        \vfill
        \item \textbf{Step 6: Calculate p-value (Two-sided):}
        Find the proportion of permuted statistics as extreme as the observed one.
        \[ p = \frac{\{|t^\ast| \ge |t_{\text{obs}}|\}}{R} \]
    \end{enumerate}
\end{frame}
  
    \begin{block}{Execution and Results (R=10,000)}
    \scriptsize
\begin{verbatim}
> cat("Observed Mean Difference (Time):", res_time$observed, "\n")
Observed Mean Difference (Time): 1.6908 

> cat("P-value (Time):", res_time$p_value, "\n")
P-value (Time): 1e-04 
\end{verbatim}
    \end{block}
\end{frame}

\begin{frame}
    \begin{figure}
        \centering
        \includegraphics[width=0.7\linewidth]{images/permutation_mean.png}
        \caption{\footnotesize Permutation Distribution: Mean TIme Difference (New - Old).}
    \end{figure}
\end{frame}

% =========================================
\section{Analysis 2: Conversion Rate}
% =========================================

\begin{frame}[fragile]{Analysis 2: Conversion Rate}
    \begin{block}{Test Statistic Function (Proportion Difference)}
    \[ t = \hat{p}_{\text{new}} - \hat{p}_{\text{old}} \]
    \scriptsize
    \end{block}
    
    \begin{block}{Execution and Results (R=10,000)}
    \scriptsize
\begin{verbatim}
> cat("Observed Proportion Difference (Conversion):", res_conv$observed, "\n")
Observed Proportion Difference (Conversion): 0.24  

> cat("P-value (Conversion):", res_conv$p_value, "\n")
P-value (Conversion): 0.0147 
\end{verbatim}
    \end{block}
\end{frame}

\begin{frame}
    \begin{figure}
        \centering
        \includegraphics[width=0.7\linewidth]{images/permutation_conver_rate.png}
        \caption{\footnotesize Permutation Distribution: Conversion Rate Difference (New - Old).}
    \end{figure}
\end{frame}

% =========================================
\section{Conclusion}
% =========================================

\begin{frame}{Interpretation and Conclusion}
    \begin{block}{Analysis 1: Time-on-Page (H1: new > old)}
        \begin{itemize}
            \item The observed p-value is \textbf{0.0001}.
            \item Since the p-value ($0.0001$) is less than our significance level ($\alpha = 0.05$), we **reject the null hypothesis ($H_0$)**.
            \item \textbf{Conclusion:} We have statistically significant evidence that users spend more time on the new landing page.
        \end{itemize}
    \end{block}
    
    \vfill
    
    \begin{block}{Analysis 2: Conversion Rate (H1: new > old)}
        \begin{itemize}
            \item The observed p-value is \textbf{0.0147}.
            \item Since the p-value ($0.0147$) is less than our significance level ($\alpha = 0.05$), we **reject the null hypothesis ($H_0$)**.
            \item \textbf{Conclusion:} We have statistically significant evidence that the new landing page has a higher conversion rate.
        \end{itemize}
    \end{block}
\end{frame}

\begin{frame}{Limitations}
    \begin{itemize}
        \item \textbf{Sample may not be representative}
        \begin{itemize}
            \item Data (e.g., 100 users) may come from a specific period, country, or device type, not reflecting the entire user base.
        \end{itemize}
        \vfill
        \item \textbf{Small sample size in each group}
        \begin{itemize}
            \item e.g., Only 50 in the Old group, 50 in the New group.
            \item The permutation distribution can be unstable (noisy).
        \end{itemize}
        \vfill
        \item \textbf{Independence Assumption}
        \begin{itemize}
            \item The permutation test assumes users are i.i.d (independent and identically distributed).
            \item It ignores clusters (e.g., the same user, same campaign). More complex methods are needed if clustering is present.
        \end{itemize}
    \end{itemize}
\end{frame}

\begin{frame}{Methodological Notes}
    \begin{itemize}
        \item The analyses utilized $R = 10,000$ permutations. This number is generally sufficient for stable p-value estimation.
        \vfill
        \item This non-parametric (permutation) approach is robust as it avoids the assumptions of normality or homoscedasticity (equal variances) required by parametric counterparts (e.g., the two-sample t-test).
        \vfill
        \item The only core assumption is the **exchangeability** of observations under the null hypothesis ($H_0$).
    \end{itemize}
\end{frame}

