\section{Case study: Improving Library User Experience with A/B Testing}

\begin{frame}
    \begin{figure}
      \centering
        \includegraphics[width=0.7\linewidth]{images/image_crazy_egg.png}
        \caption{Library Homepage Click Data - April 3-April 10, 2013}    
    \end{figure}
\end{frame}

% Slide 2: Introduction to A/B Testing
\begin{frame}
    \frametitle{Case study: Improving Library User Experience}
    
    \begin{itemize}
        \item \textbf{Problem Identified:} The homepage category 
        "\textbf{Interact}" had an extremely low 
        \textbf{2\% Click-Through Rate (CTR)}.
        \item \textbf{Research Question:} Will changing the 
        confusing category title lead to a measurable 
        increase in user engagement?
        \item \textbf{Refinement:} Used brief user interviews 
        to select the most meaningful title variations for testing.
        \item \textbf{Hypothesis:} Replacing the title with "\textbf{Help}" or "\textbf{Services}" will generate significantly higher user engagement compared to all other options.
    \end{itemize}
    
\end{frame}

% Slide 2: method
\begin{frame}
    \frametitle{Case study: Improving Library User Experience}
        \begin{itemize}
        \item \textbf{Set up and run experiment:} Users 
        were randomly served one of the five variations 
        (Control: Interact, Variations: Connect, Learn, Help, 
        Services) over a set period. 
        Tools used included Google Analytics and Crazy Egg.

      
    \end{itemize}

\end{frame}

\begin{frame}
    \begin{figure}
      \centering
        \includegraphics[width=0.7\linewidth]{images/click_through_bar.png}
        \caption{Click through rate by title variation}    
    \end{figure}
\end{frame}


%drop_off_rate =>>>> explain cho thaafy
\begin{frame}
    \begin{figure}
      \centering
        \includegraphics[width=0.7\linewidth]{images/drop_off_page.png}
        \caption{Drop off rates by title variation}    
    \end{figure}
\end{frame}

% home page return rate ==> explain cho thay
\begin{frame}
    \begin{figure}
      \centering
        \includegraphics[width=0.7\linewidth]{images/Home_page_return.png}
        \caption{Homepages return rates by title variation}    
    \end{figure}
\end{frame}

% Slide 2: Analysís and conclude.
\begin{frame}
    \frametitle{Case study: Improving Library User Experience}
    
    \begin{itemize}
        \item \textbf{Winning Variation:} The title "\textbf{Services}" was the highest-performing option across all metrics (CTR, Drop-Off, Return Rate).
        \item \textbf{Unexpected Finding:} The internally favored title, "\textbf{Learn}," generated the \textbf{lowest user engagement}.
        \item \textbf{Validation:} This confirmed 
        the value of A/B testing—relying 
        on internal opinion would have resulted in a worse UX.
    \end{itemize}
    
\end{frame}

\begin{frame}
     \frametitle{Data Collection Process Overview}
\begin{itemize}
    \item \textbf{Define research question:} Identify UX issues or engagement problems on library platforms.
    \item \textbf{Conduct qualitative interviews:} Gather user insights to refine and validate variations to test.
    \item \textbf{Formulate hypothesis and metrics:} Decide what to measure (click-through rate, drop-off rate, etc.).
    \item \textbf{Set up experiment:} Deploy A/B or A/B/n variations randomly to users with controlled sampling.
    \item \textbf{Collect and analyze data:} Track defined metrics and compare the performance of variations.
    \item \textbf{Share results and decide:} Implement the winning variation based on the analysis.
\end{itemize}
\end{frame}